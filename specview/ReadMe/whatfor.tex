\section{Introduction}
\subsection{What does it do?}
The aim is to build a viewer displaying the output of a chemical experiment:

-Mass Spectra experiment

-NMR experiment\\
Each output experiment provides the information about:

-The molecule studied.

-Its spectra.

-The additional information about the condition of the experiment\\
Hence, the viewer must \textbf{displays} the\textbf{ molecule} along with its \textbf{spectrum}. Besides, we want it to be \textbf{interactive} with the user, e.g when the user mouse over one peak, it has to \textbf{highlights} and highlights the corresponding atoms it is linked to(in the case of a \href{http://en.wikipedia.org/wiki/Nuclear\_magnetic\_resonance}{NMR experiment}) and it has to display the corresponding fragment(s) (in case of a \href{http://en.wikipedia.org/wiki/Mass\_spectrometry}{MS experiment}).
\subsection{How do we do it?}
The entire viewer is build with JavaScript and use the \href{http://code.google.com/closure/}{google closure library}

\subsubsection{Google Closure}

In summary, the Closure Library is a JavaScript library which is based on a modulable architecture. It provides cross-browser functions for DOM manipulations and events, AJAX and JSON, as well as more high-level objects such as User Interface widgets and controls.
For instance and in our case, it provides the tools to design a canvas editor, and all the objects to have a control over this canvas.
\clearpage
